% FortySecondsCV LaTeX template
% Copyright © 2019 René Wirnata <rene.wirnata@pandascience.net>
% Licensed under the 3-Clause BSD License. See LICENSE file for details.
%
% Attributions
% ------------
% * fortysecondscv is based on the twentysecondcv class by Carmine Spagnuolo 
%   (cspagnuolo@unisa.it), released under the MIT license and available under
%   https://github.com/spagnuolocarmine/TwentySecondsCurriculumVitae-LaTex
% * further attributions are indicated immediately before corresponding code


%-------------------------------------------------------------------------------
%                             ADDITIONAL PACKAGES
%-------------------------------------------------------------------------------
\documentclass[
 a4paper, 11pt,
%   showframes,
%   maincolor=cvgreen,
%   sectioncolor=red,
%   subsectioncolor=orange
  sidebarwidth=0.32\paperwidth,
%   topbottommargin=0.03\paperheight,
%   leftrightmargin=20pt
]{fortysecondscv}

% improve word spacing and hyphenation
\usepackage{microtype}
\usepackage{ragged2e}

% take care of proper font encoding
\ifxetex
  \usepackage{fontspec}
  \defaultfontfeatures{Ligatures=TeX}
  % \newfontfamily\headingfont[Path = fonts/]{segoeuib.ttf} % local font
\else
  \usepackage[utf8]{inputenc}
  \usepackage[T1]{fontenc}
  % \usepackage[sfdefault]{noto} % use noto google font
\fi

% enable mathematical syntax for some symbols like \varnothing
\usepackage{amssymb}

% bubble diagram configuration
\usepackage{smartdiagram}
\smartdiagramset{
  % defaut font size is \large, so adjust to harmonize with sidebar layout
  bubble center node font = \footnotesize,
  bubble node font = \footnotesize,
  % default: 4cm/2.5cm; make minimum diameter relative to sidebar size
  bubble center node size = 0.4\sidebartextwidth,
  bubble node size = 0.25\sidebartextwidth,
  distance center/other bubbles = 1.5em,
  % set center bubble color
  bubble center node color = maincolor!70,
  % define the list of colors usable in the diagram
  set color list = {maincolor!10, maincolor!40,
  maincolor!20, maincolor!60, maincolor!35},
  % sets the opacity at which the bubbles are shown
  bubble fill opacity = 0.8,
}

\titlespacing{\section}{0pc}{2pt}{1pt}
\titlespacing{\subsection}{0pc}{1pc}{0pc}
%-------------------------------------------------------------------------------
%                            PERSONAL INFORMATION
%-------------------------------------------------------------------------------
% profile picture
\cvprofilepic{./photo_cv.jpg}
% your name
\cvname{Clément Hottier}
% job title/career
\cvjobtitle{Doctorant Astrophysique}
% date of birth
\cvbirthday{22 Juin 1992}
% phone number
\cvphone{+33 1 45 07 79 07}
% personal website
% \cvsite{https://pandascience.net}
% email address
\cvmail{clement.hottier@obspm.fr}
% pgp key
% add additional information
% \newcommand{\additional}{some more?}


%-------------------------------------------------------------------------------
%                              SIDEBAR 1st PAGE
%-------------------------------------------------------------------------------
% overwrite default icons and order of personal information
\renewcommand{\personaltable}{%
  \begin{personal}
    \circleicon{\faInfo}     & \cvbirthday \\
    \circleicon{\faAt}       & \cvmail \\
    % \circleicon{\faGlobe}    & \cvsite \\
    \circleicon{\faPhone}    & \cvphone \\
    % add another line
  \end{personal}
}

% add more profile sections to sidebar on first page
\addtofrontsidebar{
  % include gosquare national flags from https://github.com/gosquared/flags;
  % naming according to ISO 3166-1 alpha-2 country codes
  \graphicspath{{pics/flags/}}

  \profilesection{Langue}
  \pointskill{\flag{FR.png}}{Français}{5}
  \pointskill{\flag{GB.png}}{Anglais}{4}
  \pointskill{\flag{ES.png}}{Espagnol}{1}

  \profilesection{Programmation}
  \barskill{}{Python}{85}
  \barskill{}{Java}{80}
  \barskill{}{C}{60}
  \barskill{}{Fortran}{60}
  \barskill{}{bash}{70}
  \barskill{}{slurm}{85}

}

%-------------------------------------------------------------------------------
%                         TABLE ENTRIES RIGHT COLUMN
%-------------------------------------------------------------------------------
\begin{document}

\makefrontsidebar

\cvsection{Expériences Professionnelles}
\begin{cvtable}
  \cvitem{2016-}{Doctorat}{Observatoire de Paris - Gepi}{Étude de la structure de la Voie
    Lactée
  avec les données Gaia, 2MASS, UKIDSS, VVV}
  \cvitem{Mars - Juillet 2016}{Stage M2}{Institut d'Astrophysique de Paris}
  {Modélisation non sphérique des amas de galaxies par méthode de Metropolis Astings}
  \cvitem{Avril - Juillet 2015}{Stage M1}{Observatoire de Paris - LERMA}
  {Étude de l'impact des Quasars radio loud sur le milieu intergalactique à l'époque de la
  réionisation}
\end{cvtable}


\cvsection{Formations}
\begin{cvtable}
  \cvitem{2016-}{\textbf{Doctorat}}{Observatoire de Paris - Gepi}{}
  \cvitem{2015-2016}{Master 2}{Observatoire de Paris}{Dynamique des systèmes gravitationnels}
  \cvitem{2014-2015}{Master 1}{Observatoire de Paris}{Sciences de L'univers et technologies spatiales}
  \cvitem{2011-2014}{Licence}{Université Lille 1}{Physique fondamentale}
\end{cvtable}

\cvsection{Communications}
\subsection{Publications}
\begin{cvtable}
  \cvitem{Février 2019}
  {Gaia-2MASS 3D maps of Galactic interstellar dust\\within 3 kpc}
  {\href{https://ui.adsabs.harvard.edu/\#abs/2019arXiv190204116L/abstract}{A\&A Sous Presse}}
  {Lallement \textit{et. al.}}

  \cvitem{Août 2018}
  {Imprints of quasar duty cycle on the 21-cm signal from the Epoch of Reionization}
  {\href{https://ui.adsabs.harvard.edu/\#abs/2019arXiv190204116L/abstract}{MNRAS}}
  {Bolgar \textit{et. al.}}
\end{cvtable}

\subsection{Conférences}
\begin{cvtable}
  \cvitem{Avril 2019}
  {Dust distribution in the Milky Way}
  {$53^{rd}$ ESLAB}
  {Hottier \textit{et. al.}}

  \cvitem{Juillet 2018}
  {Dust and star distribution in the Milky Way disk}
  {\href{https://ui.adsabs.harvard.edu/\#abs/2018sf2a.conf..345H/abstract}{SF2A Proceeding}}
  {Hottier \textit{et. al.}}
\end{cvtable}

\cvsection{Enseignement}
\subsection{Monitorat}
\begin{cvtable}
  \cvitem{2016-2019}{Travaux dirigés}{M1 SUTS}{Mécanique Quantique -
  Programmation Fortran}
  
  \cvitem{2016-2019}{Cours magistral}{M1 SUTS} {Programmation Fortran - Latex}

  \cvitem{2016-2018}{Tutorat à distance}{DU LU}{6 étudiants}

  \cvitem{2016-2019}{Observations astronomiques}{DU ECU - FP}{Stage à l'Observatoire de Haute
  Provence T152 - Soirée lunette Arago}
\end{cvtable}

\subsection{Médiation scientifique}
\begin{cvtable}
  \cvitem{2018}{Conférence Pint of science}{Paris}{La mission spatiale Gaia ; avec Olivier La
  Marle}
\end{cvtable}

\cvsection{Responsabilités Administratives}
\begin{cvtable}
  \cvitemshort{2017-2019}{Représentant des étudiants de l'Observatoire de
  Paris au Conseil d'Administration (CA), au conseil de l'Unité Formation-Enseignement (CUFE)
et au conseil de la documentation (CDOC)}
\end{cvtable}

\end{document} 
