% FortySecondsCV LaTeX template
% Copyright © 2019 René Wirnata <rene.wirnata@pandascience.net>
% Licensed under the 3-Clause BSD License. See LICENSE file for details.
%
% Attributions
% ------------
% * fortysecondscv is based on the twentysecondcv class by Carmine Spagnuolo 
%   (cspagnuolo@unisa.it), released under the MIT license and available under
%   https://github.com/spagnuolocarmine/TwentySecondsCurriculumVitae-LaTex
% * further attributions are indicated immediately before corresponding code


%-------------------------------------------------------------------------------
%                             ADDITIONAL PACKAGES
%-------------------------------------------------------------------------------
\documentclass[
 a4paper, 9.5pt,
%   showframes,
%   maincolor=cvgreen,
%   sectioncolor=red,
%   subsectioncolor=orange
  sidebarwidth=0.28\paperwidth,
%   topbottommargin=0.03\paperheight,
%   leftrightmargin=20pt
]{fortysecondscv}

\definecolor{linkcol}{HTML}{0066CC}
\hypersetup{colorlinks=true,
    linkcolor=blue,
    filecolor=magenta,      
    urlcolor=linkcol,
  }
% improve word spacing and hyphenation
\usepackage{microtype}
\usepackage{ragged2e}

% take care of proper font encoding
\ifxetex
  \usepackage{fontspec}
  \defaultfontfeatures{Ligatures=TeX}
  % \newfontfamily\headingfont[Path = fonts/]{segoeuib.ttf} % local font
\else
  \usepackage[utf8]{inputenc}
  \usepackage[T1]{fontenc}
  % \usepackage[sfdefault]{noto} % use noto google font
\fi

% enable mathematical syntax for some symbols like \varnothing
\usepackage{amssymb}

% bubble diagram configuration
\usepackage{smartdiagram}
\smartdiagramset{
  % defaut font size is \large, so adjust to harmonize with sidebar layout
  bubble center node font = \footnotesize,
  bubble node font = \footnotesize,
  % default: 4cm/2.5cm; make minimum diameter relative to sidebar size
  bubble center node size = 0.4\sidebartextwidth,
  bubble node size = 0.25\sidebartextwidth,
  distance center/other bubbles = 1.5em,
  % set center bubble color
  bubble center node color = maincolor!70,
  % define the list of colors usable in the diagram
  set color list = {maincolor!10, maincolor!40,
  maincolor!20, maincolor!60, maincolor!35},
  % sets the opacity at which the bubbles are shown
  bubble fill opacity = 0.8,
}

\titlespacing{\section}{0pc}{2pt}{1pt}
\titlespacing{\subsection}{0pc}{1pc}{0pc}
%-------------------------------------------------------------------------------
%                            PERSONAL INFORMATION
%-------------------------------------------------------------------------------
% profile picture
\cvprofilepic{./photo_cv.jpg}
% your name
\cvname{Clément\\Hottier}
% job title/career
\cvjobtitle{Post doctorant}
% date of birth
\cvbirthday{22 Juin 1992}
% phone number
\cvphone{+33 1 45 07 79 07}
% personal website
% \cvsite{https://pandascience.net}
% email address
\cvmail{clement.hottier@obspm.fr}
% pgp key
% add additional information
% \newcommand{\additional}{some more?}


%-------------------------------------------------------------------------------
%                              SIDEBAR 1st PAGE
%-------------------------------------------------------------------------------
% overwrite default icons and order of personal information
\renewcommand{\personaltable}{%
  \begin{personal}
    \circleicon{\faInfo}     & \cvbirthday \\
    \circleicon{\faAt}       & \cvmail \\
    % \circleicon{\faGlobe}    & \cvsite \\
    \circleicon{\faPhone}    & \cvphone \\
    % add another line
  \end{personal}
}

% add more profile sections to sidebar on first page
\addtofrontsidebar{
  % include gosquare national flags from https://github.com/gosquared/flags;
  % naming according to ISO 3166-1 alpha-2 country codes
  \graphicspath{{pics/flags/}}

  \profilesection{Langue}
  \pointskill{\flag{FR.png}}{Français}{5}
  \pointskill{\flag{GB.png}}{Anglais}{4}
  \pointskill{\flag{ES.png}}{Espagnol}{1}

  \profilesection{Programmation}
  \barskill{}{ADQL/SQL}{85}
  \barskill{}{Python/numpy/scipy/pandas}{95}
  \barskill{}{Java}{90}
  \barskill{}{Fortran/C}{70}
  \barskill{}{Bash}{85}
  \barskill{}{Cluster de calculs (slurm)}{95}
  \barskill{}{Docker}{65}

}

%-------------------------------------------------------------------------------
%                         TABLE ENTRIES RIGHT COLUMN
%-------------------------------------------------------------------------------
\begin{document}

\makefrontsidebar

\cvsection{Expériences de recherche}
\begin{cvtable}
  \cvitem{2021-2023}{Post Doctorant}{Observatoire de Paris - GEPI}{Tomographie cinématique du milieu
  interstellaire}
  \cvitem{2019-2020}{ATER}{Observatoire de Paris - GEPI- UFE}{Étude de la structure de la Voie
    lactée
  avec les données Gaia, 2MASS, UKIDSS, VVV}
  % \cvitem{Mars - Juillet 2016}{Stage M2}{Institut d'Astrophysique de Paris}
  % {Modélisation non sphérique des amas de galaxies par méthode de Metropolis Hastings}
  % \cvitem{Avril - Juillet 2015}{Stage M1}{Observatoire de Paris - LERMA}
  % {Étude de l'impact des Quasars radio loud sur le milieu intergalactique à l'époque de la
  % réionisation}
\end{cvtable}


\cvsection{Formations}
\begin{cvtable}
  \cvitem{2016-2019}{Doctorat}{Observatoire de Paris - Gepi}{Distribution 3D d'étoiles et de
  poussière dans le disque de la Voie lactée}
  \cvitem{2014-2016}{Master}{Observatoire de Paris}{Dynamique des systèmes gravitationnels}
  % \cvitem{2014-2015}{Master 1}{Observatoire de Paris}{Sciences de L'univers et technologies spatiales}
  % \cvitem{2011-2014}{Licence}{Université Lille 1}{Physique fondamentale}
\end{cvtable}

\cvsection{Communications}
\subsection{Publications}
\begin{cvtable}
  \cvitem{2021}
  {FEDReD III: Unraveling the 3D structure \\of Vela and Puppis}
  {Soumit A\&A}
  {Hottier \textit{et.al.}}

  \cvitem{2020}
  {FEDReD II: 3D Extinction Map\\ with 2MASS and Gaia DR2 data}
  {\href{https://ui.adsabs.harvard.edu/abs/2020arXiv200703734H/abstract}{A\&A}}
  {Hottier \textit{et.al.}}

  \cvitem{2020}
  {FEDReD I: 3D dust and stellar maps by \\Bayesian deconvolution}
  {\href{https://ui.adsabs.harvard.edu/abs/2020arXiv200704455B/abstract}{A\&A}}
  {Babusiaux \textit{et.al.}}

  \cvitem{2019}
  {Gaia-2MASS 3D maps of Galactic interstellar dust\\within 3 kpc}
  {\href{https://ui.adsabs.harvard.edu/\#abs/2019arXiv190204116L/abstract}{A\&A}}
  {Lallement \textit{et. al.}}

  \cvitem{2018}
  {Imprints of quasar duty cycle on the 21-cm signal from the Epoch of Reionization}
  {\href{https://ui.adsabs.harvard.edu/\#abs/2019arXiv190204116L/abstract}{MNRAS}}
  {Bolgar \textit{et. al.}}
\end{cvtable}


\subsection{Conférences}
\begin{cvtable}
  \cvitem{2020}{Unravelling the Vela
  clouds}{\href{https://sftools-bigdata.sciencesconf.org/336375}{SFtools-bigdata}}
  {}% {Hottier \textit{et. al.}}

  \cvitem{2019}
  {Dust distribution in the Milky Way}
  {\href{https://ui.adsabs.harvard.edu/abs/2019gaia.confE..55H/abstract}{$53^{rd}$ ESLAB}}
  {} % {Hottier \textit{et. al.}}

  \cvitem{2018}
  {Dust and star distribution in the Milky Way disk}
  {\href{https://ui.adsabs.harvard.edu/\#abs/2018sf2a.conf..345H/abstract}{SF2A Proceeding}}
  {}% {Hottier \textit{et. al.}}

\end{cvtable}

\cvsection{Enseignement}
\subsection{Université}
\begin{cvtable}
  \cvitem{2020-2021}{Stage d'observation M2}{Obs. Paris}{Stage à l'Observatoire de Haute
  Provence T152}

  \cvitem{2016-2019}{Travaux dirigés}{M1 Obs. Paris}{Mécanique Quantique -
  Programmation Fortran}
  
  \cvitem{2016-2020}{Cours magistral}{M1 Obs. Paris} {Programmation Fortran - Latex}

  \cvitem{2016-2020}{Tutorat à distance}{DU Obs. Paris}{12 étudiants}

  \cvitem{2016-2020}{Observations astronomiques}{DU/FP/ Obs. Paris}{Stage à l'Observatoire de Haute
  Provence T152 - Soirée lunette Arago}
\end{cvtable}

\subsection{Médiation scientifique}
\begin{cvtable}
  \cvitem{2020}{Fête de la science à
  l'Observatoire}{\href{https://youtu.be/r70a8Ztte4w}{Youtube Live}}{Gaia : Le plus vaste
catalogue d'étoiles}
  \cvitem{2019}{Conférence étoiles en
  scène}{\href{https://www.obspm.fr/etoiles-en-scene-au-grand-rex.html}{Com Obs. Paris}}{Conférence express au Grand Rex}
  \cvitem{2018}
  {\textbf{La mission spatiale Gaia}}
  {\href{https://pintofscience.fr/}{Pint of science France}}{}
\end{cvtable}

\cvsection{Responsabilités Administratives}
\begin{cvtable}

  \cvitemshort{2019}{Membre du comité de pilotage pour la réforme du Master 1 SUTS de
l'Observatoire de Paris}

  \cvitemshort{2017-2019}{Représentant des étudiants de l'Observatoire de
  Paris au Conseil d'administration (CA), au conseil de l'Unité Formation-Enseignement (CUFE)
et au conseil de la documentation (CDOC)}
\end{cvtable}

\end{document} 
