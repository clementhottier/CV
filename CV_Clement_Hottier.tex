\documentclass{moderncv}
\moderncvtheme[purple]{classic}
\usepackage[margin=1.cm]{geometry}
\usepackage[french]{babel}
\usepackage[utf8]{inputenc}
\usepackage{fontenc}
\usepackage{comment}



\firstname{Cl\'ement}
\familyname{Hottier}
\address{Observatoire de Paris\\GEPI - Bat 11\\5 Place Jules Janssen}{92190 Meudon}
\email{clement.hottier@obspm.fr}
\phone{+33 1 45 07 79 07}
\title{Doctorat $3^{e}$ année}
\photo[75pt]{./photo_cv.jpg}

\begin{document}
\maketitle


\section{Formation}
\cventry{2016-2019}{Doctorat}{Observatoire de Paris ED 127}{GEPI}
{sous la direction de C. Babusiaux}{\'Etudes de la structure de la Voie Lactée à l'aide des
  données GAIA}
  \cventry{2015-2016}{Master 2 Dynamique des Systèmes Gravitationnelles}{Observatoire de
    Paris}{}{}{Validé mention Bien}
	\cventry{2014-2015}{Master 1 Sciences de l'Univers et Technologies Spatiales}{Observatoire de Paris}{}{}{Valid\'e mention Bien, class\'e 3$^e$}
	
\section{Publications}%
\cventry{02/2019}
{Gaia-2MASS 3D maps of Galactic interstellar dust within 3 kpc}
{Lallement \textit{et. al.}}
{Accepté A\&A}
{\href{https://ui.adsabs.harvard.edu/\#abs/2019arXiv190204116L/abstract}{2019arXiv190204116L}}
{}

\cventry{08/2018}
{Imprints of quasar duty cycle on the 21-cm signal from the Epoch of Reionization}
{Bolgar \textit{et. al.}}
{MNRAS}
{\href{https://ui.adsabs.harvard.edu/\#abs/2019arXiv190204116L/abstract}{2018MNRAS.478.5564B}}
{}

\section{Conférence}
\cventry{Avril 2019}
{$53^{rd}$ ESLAB symposium: the Gaia Universe}
{Noordwijk}
{Pays-Bas}
{Dust distribution in the Milky Way}
{}

\cventry{Juillet 2018}
{Journée de la SF2A}
{Bordeau}
{France}
{Dust and star distribution in the Milky Way disk}
{\href{https://ui.adsabs.harvard.edu/\#abs/2018sf2a.conf..345H/abstract}{2018sf2a.conf..345H}}

	
\section{Stages}
  \cventry{Mars - Juillet 2016}{IAP}{Paris}{Sous la direction de Gary Mamon}{}{Modélisation non sphérique d'amas de galaxies :\\
    \`A partir de l'observation de positions projetées sur le ciel et des vitesses radiales de
  traceurs dans l'amas, reconstruction des profils de masse et d'anisotropie des vitesses par
maximisation de vraisemblances (Algorithme Metropolis-Asting).}
  
	\cventry{Avril - Juillet 2015}{LERMA}{Observatoire de Paris}{Sous la direction de Benoit
  Semelin}{}{\'Etude de l'impact des Quasar raio loud sur le MIG}

\section{Informatique}
	\cvdoubleitem {Langages } 			{C}	{}	{Fortran 90}
	\cvdoubleitem {}					{Python}		{} 	{Java}
	\cvdoubleitem {}					{Bash}	{}	{LaTex}

  \cvdoubleitem {Calcul Parallèle} {OpenMP} {} {MPI}
  \cvdoubleitem {} {Java} {} {Python}
	 
	
\section{Langues}
\cvitem{Français}{Langue Maternelle}
	\cvitem{Anglais}{B2 écrit, B2 parlé}
	\cvitem{Espagnol}{A2 écrit, A2 parlé} 

\end{document}
