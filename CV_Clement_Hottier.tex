% FortySecondsCV LaTeX template
% Copyright © 2019 René Wirnata <rene.wirnata@pandascience.net>
% Licensed under the 3-Clause BSD License. See LICENSE file for details.
%
% Attributions
% ------------
% * fortysecondscv is based on the twentysecondcv class by Carmine Spagnuolo 
%   (cspagnuolo@unisa.it), released under the MIT license and available under
%   https://github.com/spagnuolocarmine/TwentySecondsCurriculumVitae-LaTex
% * further attributions are indicated immediately before corresponding code


%-------------------------------------------------------------------------------
%                             ADDITIONAL PACKAGES
%-------------------------------------------------------------------------------
\documentclass[
a4paper, 
%   showframes,
%   maincolor=cvgreen,
%   sectioncolor=red,
%   subsectioncolor=orange
%   sidebarwidth=0.4\paperwidth,
%   topbottommargin=0.03\paperheight,
%   leftrightmargin=20pt
]{fortysecondscv}

% improve word spacing and hyphenation
\usepackage{microtype}
\usepackage{ragged2e}

% take care of proper font encoding
\ifxetex
  \usepackage{fontspec}
  \defaultfontfeatures{Ligatures=TeX}
  % \newfontfamily\headingfont[Path = fonts/]{segoeuib.ttf} % local font
\else
  \usepackage[utf8]{inputenc}
  \usepackage[T1]{fontenc}
  % \usepackage[sfdefault]{noto} % use noto google font
\fi

% enable mathematical syntax for some symbols like \varnothing
\usepackage{amssymb}

% bubble diagram configuration
\usepackage{smartdiagram}
\smartdiagramset{
  % defaut font size is \large, so adjust to harmonize with sidebar layout
  bubble center node font = \footnotesize,
  bubble node font = \footnotesize,
  % default: 4cm/2.5cm; make minimum diameter relative to sidebar size
  bubble center node size = 0.4\sidebartextwidth,
  bubble node size = 0.25\sidebartextwidth,
  distance center/other bubbles = 1.5em,
  % set center bubble color
  bubble center node color = maincolor!70,
  % define the list of colors usable in the diagram
  set color list = {maincolor!10, maincolor!40,
  maincolor!20, maincolor!60, maincolor!35},
  % sets the opacity at which the bubbles are shown
  bubble fill opacity = 0.8,
}


%-------------------------------------------------------------------------------
%                            PERSONAL INFORMATION
%-------------------------------------------------------------------------------
% profile picture
\cvprofilepic{pics/profile.png}
% your name
\cvname{Clément Hottier}
% job title/career
\cvjobtitle{Doctorant Astrophysique}
% date of birth
\cvbirthday{22 Juin 1992}
% phone number
\cvphone{+33 6 60 35 56 85}
% personal website
\cvsite{https://pandascience.net}
% email address
\cvmail{clement.hottier@obspm.fr}
% pgp key
% add additional information
% \newcommand{\additional}{some more?}


%-------------------------------------------------------------------------------
%                              SIDEBAR 1st PAGE
%-------------------------------------------------------------------------------
% overwrite default icons and order of personal information
\renewcommand{\personaltable}{%
  \begin{personal}
    \circleicon{\faAt}       & \cvmail \\
    \circleicon{\faGlobe}    & \cvsite \\
    \circleicon{\faPhone}    & \cvphone \\
    \circleicon{\faInfo}     & \cvbirthday \\
    % add another line
  \end{personal}
}

% add more profile sections to sidebar on first page
\addtofrontsidebar{
  % include gosquare national flags from https://github.com/gosquared/flags;
  % naming according to ISO 3166-1 alpha-2 country codes
  \graphicspath{{pics/flags/}}

  \profilesection{Langue}
  \pointskill{\flag{FR.png}}{Français}{5}
  \pointskill{\flag{GB.png}}{Anglais}{4}
  \pointskill{\flag{ES.png}}{Espagnol}{1}

  \profilesection{Programmation}
  \barskill{}{Python}{85}
  \barskill{}{Java}{80}
  \barskill{}{C}{60}
  \barskill{}{Fortran}{60}
  \barskill{}{bash}{70}
  \barskill{}{slurm}{85}

}


%-------------------------------------------------------------------------------
%                              SIDEBAR 2nd PAGE
%-------------------------------------------------------------------------------
\addtobacksidebar{
  \profilesection{About Me}
  \aboutme{
    The giant panda is a terrestrial animal and primarily spends its life
    roaming and feeding in the bamboo forests of the Qinling Mountains and in
    the hilly province of Sichuan.
  }

  \profilesection{Diagrams}
  \chartlabel{Bubble Diagram}
  \begin{figure}\centering
    \smartdiagram[bubble diagram]{
      \textcolor{white}{\textbf{Being a}} \\ 
      \textcolor{white}{\textbf{Panda}}, % center bubble	
      \textcolor{black!90}{Eating},
      \textcolor{black!90}{Sleeping},
      \textcolor{black!90}{Rolling},
      \textcolor{black!90}{Playing},
      \textcolor{black!90}{Chilling}
    }
  \end{figure}

  \chartlabel{Wheel Chart}

  \wheelchart{4em}{2em}{%
    20/3em/maincolor!50/Chill,
    15/3em/maincolor!15/Play,
    30/4em/maincolor!40/Sleep,
    20/3em/maincolor!20/Eat
  }

  \profilesection{Barskills}
  \barskill{\faSkyatlas}{Wearing asian rice hats}{60}
  \barskill{\faImage}{Playing Chess}{30}
  \barskill{\faMusic}{Playing the bamboo flute}{50}

  \profilesection{Memberships}
  \begin{memberships}
    \membership{pics/logo.png}{PandaScience.net}
    \membership{pics/logo.png}{Here's some longer text spanning over more than
    only one line}
  \end{memberships}

}


%-------------------------------------------------------------------------------
%                         TABLE ENTRIES RIGHT COLUMN
%-------------------------------------------------------------------------------
\begin{document}

\makefrontsidebar

\cvsection{Expérience Professionnelle}
\begin{cvtable}
  \cvitem{2016-}{Doctorat}{Observatoire de Paris - Gepi}{Étude de la structure de la Voie Lacté
  avec les données Gaia, 2MASS, UKIDSS, VVV}
  \cvitem{Mars - Juillet 2016}{Stage M2}{Institut d'Astrophysique de Paris}
  {Modélisation non sphérique des amas de galaxies par méthode de Metropolis Astings}
  \cvitem{Avril - Juillet 2015}{Stage M1}{Observatoire de Paris - LERMA}
  {Étude de l'impact des Quasar radio Loud sur milieu intergalactique à l'époque de la
  réionisation}
\end{cvtable}


\cvsection{Formation}
\begin{cvtable}
  \cvitem{2016-}{Doctorat}{Observatoire de Paris - Gepi}{}
  \cvitem{2015-2016}{Master 2 Dynamique des systèmes gravitationnelle}{Observatoire de Paris}{}
  \cvitem{2014-2016}{Master 1 Science de L'univers et technologies spatiales}{Observatoire de Paris}{}
  \cvitem{2011-2014}{Licence Physique fondamentale}{Université Lille 1}{}
\end{cvtable}

\cvsection{Publications}
\begin{cvtable}
  \cvitem{2010}{Cooking: 100 recipes for lazy Pandas}{Panda's
  Culinary World}{}
  \cvitem{2005}{Pandastasia}{Bamboo Books Assoc.}{}
  \cvitem{2000}{The Panda Way - A guide for mastering everyday life as a Panda}
  {Young Panda's Journal}{}
\end{cvtable}


\cvsection{Awards}
\begin{cvtable}
  \cvitem{2010 -- now}{Panda of the Year}{Panda World Forum}{}
  \cvitem{2005 -- now}{Face of World Wide Fund for Nature}{WWF}{}
  \cvitem{2000}{Winner of Bamboo Sprouts Eating Contest}{Bamboo Society}{}
\end{cvtable}


\cvsection{Extra-Curricular Activities}
\begin{cvtable}
  \cvitemshort{Relaxing}{Master the fine art of relaxing everywhere}
  \cvitemshort{Music}{Playing the bamboo flute in the 1st Panda Orchestra}
  \cvitemshort{Education}{Teaching young pandas to be more panda-like}
\end{cvtable}


\newpage
\makebacksidebar


\cvsection{section}
\subsection{Subsection}
\begin{cvtable}
  \cvitem{<dates>}{<cv-item title>}{<location>}{<optional: description>}
\end{cvtable}

\cvsection{cvitem}
\subsection{Multi-line with longer description}
\begin{cvtable}
  \cvitem{date}{Description}{location}{Some longer and more detailed 
  description, that takes two lines	of space instead of only one.}
  \cvitem{date}{Description}{location}{Some longer and more detailed 
  description, that takes two lines	of space instead of only one.}
  \cvitem{date}{Description}{location}{Some longer and more detailed 
  description, that takes two lines	of space instead of only one.}
\end{cvtable}

\subsection{One-line without description}
\begin{cvtable}
  \cvitem{Award}{One-line description}{Sponsor}{}
  \cvitem{Award}{One-line description}{Sponsor}{}
  \cvitem{Award}{One-line description}{Sponsor}{}
\end{cvtable}

\cvsection{cvitemshort}
\subsection{One-line}
\begin{cvtable}
  \cvitemshort{Key}{Some further description}
  \cvitemshort{Key}{Some further description}
  \cvitemshort{Key}{Some further description}
\end{cvtable}

\subsection{Multi-line with longer description}
\begin{cvtable}
  \cvitemshort{Key}{Some further description. Can fill even more than
  only one single line while still keeping the correct indendation level.}
  \cvitemshort{Key}{Some further description. Can fill even more than
  only one single line while still keeping the correct indendation level.}
  \cvitemshort{Key}{Some further description. Can fill even more than
  only one single line while still keeping the correct indendation level.}
\end{cvtable}

\cvsignature

\end{document} 
