% FortySecondsCV LaTeX template
% Copyright © 2019 René Wirnata <rene.wirnata@pandascience.net>
% Licensed under the 3-Clause BSD License. See LICENSE file for details.
%
% Attributions
% ------------
% * fortysecondscv is based on the twentysecondcv class by Carmine Spagnuolo 
%   (cspagnuolo@unisa.it), released under the MIT license and available under
%   https://github.com/spagnuolocarmine/TwentySecondsCurriculumVitae-LaTex
% * further attributions are indicated immediately before corresponding code


%-------------------------------------------------------------------------------
%                             ADDITIONAL PACKAGES
%-------------------------------------------------------------------------------
\documentclass[
 a4paper, 10.5pt,
%   showframes,
%   maincolor=cvgreen,
%   sectioncolor=red,
%   subsectioncolor=orange
  sidebarwidth=0.32\paperwidth,
%   topbottommargin=0.03\paperheight,
%   leftrightmargin=20pt
]{fortysecondscv}

\definecolor{linkcol}{HTML}{0066CC}
\hypersetup{colorlinks=true,
    linkcolor=blue,
    filecolor=magenta,      
    urlcolor=linkcol,
  }

% improve word spacing and hyphenation
\usepackage{microtype}
\usepackage{ragged2e}
\usepackage{enumitem}

% take care of proper font encoding
\ifxetex
  \usepackage{fontspec}
  \defaultfontfeatures{Ligatures=TeX}
  % \newfontfamily\headingfont[Path = fonts/]{segoeuib.ttf} % local font
\else
  \usepackage[utf8]{inputenc}
  \usepackage[T1]{fontenc}
  % \usepackage[sfdefault]{noto} % use noto google font
\fi

% enable mathematical syntax for some symbols like \varnothing
\usepackage{amssymb}

% bubble diagram configuration
\usepackage{smartdiagram}
\smartdiagramset{
  % defaut font size is \large, so adjust to harmonize with sidebar layout
  bubble center node font = \footnotesize,
  bubble node font = \footnotesize,
  % default: 4cm/2.5cm; make minimum diameter relative to sidebar size
  bubble center node size = 0.4\sidebartextwidth,
  bubble node size = 0.25\sidebartextwidth,
  distance center/other bubbles = 1.5em,
  % set center bubble color
  bubble center node color = maincolor!70,
  % define the list of colors usable in the diagram
  set color list = {maincolor!10, maincolor!40,
  maincolor!20, maincolor!60, maincolor!35},
  % sets the opacity at which the bubbles are shown
  bubble fill opacity = 0.8,
}

\titlespacing{\section}{0pc}{2pt}{1pt}
\titlespacing{\subsection}{0pc}{1pc}{0pc}
%-------------------------------------------------------------------------------
%                            PERSONAL INFORMATION
%-------------------------------------------------------------------------------
% profile picture
\cvprofilepic{./photo_cv.jpg}
% your name
\cvname{Clément Hottier}
% job title/career
\cvjobtitle{Data Scientist}
% date of birth
\cvbirthday{22 Juin 1992}
% phone number
\cvphone{+33 6 60 35 56 85}
% personal website
% \cvsite{https://pandascience.net}
% email address
% \cvmail{clement.hottier@obspm.fr}
\cvmail{hottierclement@gmail.com}
% pgp key
% add additional information
% \newcommand{\additional}{some more?}


%-------------------------------------------------------------------------------
%                              SIDEBAR 1st PAGE
%-------------------------------------------------------------------------------
% overwrite default icons and order of personal information
\renewcommand{\personaltable}{%
  \begin{personal}
    \circleicon{\faInfo}     & \cvbirthday \\
    \circleicon{\faAt}       & \cvmail \\
    % \circleicon{\faGlobe}    & \cvsite \\
    \circleicon{\faPhone}    & \cvphone \\
    % add another line
  \end{personal}
}

% add more profile sections to sidebar on first page
\addtofrontsidebar{
  % include gosquare national flags from https://github.com/gosquared/flags;
  % naming according to ISO 3166-1 alpha-2 country codes
  \graphicspath{{pics/flags/}}

  \profilesection{Langue}
  \pointskill{\flag{FR.png}}{Français}{5}
  \pointskill{\flag{GB.png}}{Anglais}{4}
  \pointskill{\flag{ES.png}}{Espagnol}{1}

  \profilesection{Programmation}
  \barskill{\faIcon{database}}{ADQL/SQL}{85}
  \barskill{\faPython}{Python : numpy/scipy/pandas}{95}
  \barskill{\faCoffee}{Java}{90}
  \barskill{}{Fortran/C}{70}
  \barskill{\faIcon{linux}}{Bash}{85}
  \barskill{}{Cluster computing (slurm)}{95}
  \barskill{\faIcon{code-branch}}{Git/SVN}{75}
  \barskill{\faDocker}{Docker}{60}

}

\addtobacksidebar{
	\profilesection{Centre d'intérêts}
  \skill{}{Micro ordinateur et controleur}
  \skill{}{Drone FPV}
  \skill{}{Trek et Randonnée}
  \skill{}{Couture}
  \skill{}{Cuisine}
}
%-------------------------------------------------------------------------------
%                         TABLE ENTRIES RIGHT COLUMN
%-------------------------------------------------------------------------------
\begin{document}

\makefrontsidebar

\cvsection{Expériences Professionnelles}
\begin{cvtable}
  \cvitem{2021-2022}{Observatoire de Paris - CNES}{}{
% Conception, implémentation, test et documentation d'une librairie d'exploration et de visualisation de cube densité adapté à l'astronomie (python3 OO : matplotlib/scipy/numpy/h5py/healpy)}
    Conception d'une bibliothèque d'exploration de données 3D:
    \begin{itemize}[noitemsep,nolistsep,leftmargin=*]
      \item Implémentation en python3 orienté objet en utilisant matplotlib, scipy, numpy
      \item Optimisation mémoire et implémentation d'un "lazy mode" pyh5
      \item implémentation de la prise charge des healpix 
    \end{itemize}
    Conception d'algorithme de
  tomographie 3D (python3 scipy,numpy)}
  
  \cvitem{2019-2020}{Observatoire de Paris-UFE/GEPI}{}{
    Exploitation de cube 3D de densité :
    \begin{itemize}[noitemsep,nolistsep,leftmargin=*]
      \item Administration des cube de données brute en HDF5 (python3:h5py et Java:JHDF5)
      \item Implémentation d'un algorithme de segmentation en Python 3 (orienté objet, numpy,
        scipy, numba)
      \item Mise a disposition de données réduite et catalogue (ADQL, Table fits, Table VO)
    \end{itemize}
    % - Exploration de données et identification croisée de structures (Python 3 :
    % matplotlib/pandas)\\
    % - Mise a disposition de base de données (ADQL, Table fits, Table VO)\\
    % - Administration de données brute (HDF5 ; python3:h5py et Java:JHDF5)\\
    % - Implémentation d'algorithme de classification et création de catalogues (Python3 OO : numpy/scipy/numba)\\
    % - Création de la boite à outils de réduction d'image CCD et spectroscopie pour stage d'observation à l'OHP (python3 : astropy/numpy/matplotlib) Documentation (markdown) Distribution de la librairie en image docker et paquet python.
    Développement de la chaine de traitement de données pour les stage OHP (python3):
    \begin{itemize}[noitemsep,nolistsep,leftmargin=*]
      \item Réduction des images CCD (bias, PLU) et extraction de la photométrie 
      \item Réduction spectroscopique (bias, PLU) et automatisation de la calibration en
        longueur d'onde
      \item Documentation et cookbook en markdown et jupyter notebook
      \item Distribution sous la forme d'image docker
    \end{itemize}
    }
    
    \cvitem{2016-2019}{Observatoire de Paris - GEPI (Doctorat)}{}{
      Mise au point d'un Logiciel d'analyse et modélisation de photométrie et d'astrométrie par
      déconvolution bayésienne :
      \begin{itemize}[noitemsep,nolistsep,leftmargin=*]
        \item Rédaction d'une documentation, en Javadoc, spécifiant les besoins et les
          algorithmes 
        \item Écriture de l'algorithme de déconvolution en Java (orienté objet), passage de
          Ant/Ivy à Gradle pour la compilation et la gestion des dépendance, et de SVN à Git
          pour la gestion de version et la collaboration
        \item Analyse et optimisation du code (Yourkit), évolution des algorithmes avec le
          multithreading de Java, simplification du code la programmation
          fonctionnelle de Java 8 (temps CPU/10)
        \item Changement des formats de données, choix de HDF5 pour sa rapidité d'accès et sa
          compatibilité avec Java (JHDF5) et Python 3. Choix d'une base SQLite 3 pour
          l'indexation des résultats (vitesse accées x100, occupation x0.5)
          \item Écriture des tests unitaire et non-régression (JUnit), automatisation des tests
            et génération de la documentation avec GitlabCI, Gitlab Pages et docker.
          \item Utilisation d'un Mesocentre et de SLURM pour test des versions et application
            de la version stable (400 000 hCPU)
          \item Écriture du post-processing et de la visualisation en Python 3   avec  Numpy,
            Scipy et Matplotlib (versionnage avec Git, tests unitaires avec Green).
        \end{itemize}
      % - Conception et validation d'un algorithme d'analyse photométrique et astrométrique par déconvolution Bayésienne (Java OO, ant, graddle, svn, git)\\
      % - Documentation et tests unitaires (Javadoc, Junit)
      % - Test et génération de la documentation automatique (gitlabCI, gitlatb Pages, docker, graddle)\\
      % - pProfiling et optimisation logiciel(Java yourkit; Java multithreading ; Java : programmation fonctionnelle)\\
      % - Optimisation accées mémoire et reorganisation de data model (HDF5 ; Java : JHDF5, Java : healpix)\\
      % - Utilisation massive de grappe de calcul (slurm)\\
    % - Gestion et indexation de résultats brut dans des centre de calcul (python3, Java, sqlite)\\
  % - Développement d'un Pipeline de post processing et de visualisation (Python3 OO : numpy/scipy/matplotlib)
    }

  \cvitem{Mars - Juillet 2016}{Institut d'Astrophysique de Paris}{}
  {Conception et mise au point d'un modèle d'analyse par maximisation de vraissemblance et
  chaine de Markov (Fortran ; openMPI ; python3 : matplotlib)}
  % \cvitem{Avril - Juillet 2015}{Stage M1}{Observatoire de Paris - LERMA}
  % {Étude de l'impact des Quasars radio loud sur le milieu intergalactique à l'époque de la
  % réionisation}
\end{cvtable}


\cvsection{Responsabilités Administratives et Structurantes}
\begin{cvtable}
  \cvitemshort{2019}{Membre du comité de pilotage pour la réforme du Master 1 SUTS de
l'Observatoire de Paris : 
\begin{itemize}[noitemsep,nolistsep,leftmargin=*]
  \item Définition des nouveau enseignement et programme
    \item audition et choix des enseignant
      \item bouclage de la plaquette pour la rentré 2020
\end{itemize}
}
  \cvitemshort{2017-2019}{Représentant des étudiants de l'Observatoire de
  Paris au conseil d'administration (CA), au conseil de l'Unité Formation-Enseignement (CUFE)
et au conseil de la documentation (CDOC)}
\end{cvtable}

\newpage
\makebacksidebar

\cvsection{Formations}
\begin{cvtable}
\cvitem{2021}{\textbf{Machine learning en python}}{CNRS - ifsem}{}
\cvitem{2018}{\textbf{Docker}}{Observatoire de Paris - IN2P3}{}
\cvitem{2017}{\textbf{Python Avancé}}{Observatoire de Paris}{}
  \cvitem{2016-2019}{\textbf{Doctorat} (\href{https://tel.archives-ouvertes.fr/tel-02879238/document}{Manuscrit
  en ligne})}{Observatoire de Paris}{}
  \cvitem{2014-2016}{Master}{Observatoire de Paris}{Dynamique des systèmes gravitationnels}
  % \cvitem{2014-2015}{Master 1}{Observatoire de Paris}{Sciences de L'univers et technologies spatiales}
  % \cvitem{2011-2014}{Licence}{Université Lille 1}{Physique fondamentale}
\end{cvtable}

\cvsection{Communications}
\subsection{Publications}
\begin{cvtable}
  \cvitem{2020}
  {FEDReD II : 3D Extinction Map with 2MASS and Gaia DR2 data}
  {\href{https://ui.adsabs.harvard.edu/abs/2020arXiv200703734H/abstract}{\underline{A\&A}}}
  {}

  \cvitem{2020}
  {FEDReD I: 3D dust and stellar maps by \\Bayesian deconvolution}
  {\href{https://ui.adsabs.harvard.edu/abs/2020arXiv200704455B/abstract}{\underline{A\&A}}}
  {}
\end{cvtable}

\subsection{Conférences Internationales}
\begin{cvtable}
  \cvitem{Avril 2019}
  {Dust distribution in the Milky Way}
  {\href{https://ui.adsabs.harvard.edu/abs/2019gaia.confE..55H/abstract}{\underline{$53^{rd}$ ESLAB}}}
  {}

  \cvitem{Juillet 2018}
  {Dust and star distribution in the Milky Way disk}
  {\href{https://ui.adsabs.harvard.edu/\#abs/2018sf2a.conf..345H/abstract}{\underline{SF2A Proceeding}}}
  {}
\end{cvtable}

\cvsection{Expériences d'enseignement}
\subsection{Universitaire}
\begin{cvtable}
  \cvitem{2016-2020}{Cours Magistraux et Travaux dirigés}{Observatoire de Paris}{
  Programmation Scientifique - Programmation Latex - Mécanique Quantique}
  
  \cvitem{2016-2020}{Stage d'observation}{Observatoire de Paris/Univ Paris-Saclay}
  {- Encadrement d'observations sur instruments professionnels\\
   - Développement de la boîte à outils de réduction des données et distribution en package
 python et image Docker}
\end{cvtable}

\subsection{Médiation scientifique}
\begin{cvtable}
  \cvitem{2020}{\textbf{Conférence de Presse Gaia EDR3}}{\href{https://youtu.be/kOLjv-9lsmA?t=2145}
    {\underline{Com Obs.Paris}}}{}
  \cvitem{2020}{Fête de la science à
  l'Observatoire}{\href{https://youtu.be/r70a8Ztte4w}{\underline{Youtube Live}}}{Gaia : Le plus vaste
catalogue d'étoiles}
  \cvitem{2019}{Conférence étoiles en
  scène}{\href{https://www.obspm.fr/etoiles-en-scene-au-grand-rex.html}{\underline{Com Obs. Paris}}}{Conférence express au Grand Rex}
  \cvitem{2018}
  {Conférence Pint of science}
  {\href{https://pintofscience.fr/}{\underline{Pint of science France}}}{La mission spatiale Gaia}
\end{cvtable}

\end{document}% FortySecondsCV LaTeX template
