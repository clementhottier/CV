% FortySecondsCV LaTeX template
% Copyright © 2019 René Wirnata <rene.wirnata@pandascience.net>
% Licensed under the 3-Clause BSD License. See LICENSE file for details.
%
% Attributions
% ------------
% * fortysecondscv is based on the twentysecondcv class by Carmine Spagnuolo 
%   (cspagnuolo@unisa.it), released under the MIT license and available under
%   https://github.com/spagnuolocarmine/TwentySecondsCurriculumVitae-LaTex
% * further attributions are indicated immediately before corresponding code


%-------------------------------------------------------------------------------
%                             ADDITIONAL PACKAGES
%-------------------------------------------------------------------------------
\documentclass[
 a4paper, 10.5pt,
%   showframes,
%   maincolor=cvgreen,
%   sectioncolor=red,
%   subsectioncolor=orange
  sidebarwidth=0.32\paperwidth,
%   topbottommargin=0.03\paperheight,
%   leftrightmargin=20pt
]{fortysecondscv}

\definecolor{linkcol}{HTML}{0066CC}
\hypersetup{colorlinks=true,
    linkcolor=blue,
    filecolor=magenta,      
    urlcolor=linkcol,
  }

% improve word spacing and hyphenation
\usepackage{microtype}
\usepackage{ragged2e}

% take care of proper font encoding
\ifxetex
  \usepackage{fontspec}
  \defaultfontfeatures{Ligatures=TeX}
  % \newfontfamily\headingfont[Path = fonts/]{segoeuib.ttf} % local font
\else
  \usepackage[utf8]{inputenc}
  \usepackage[T1]{fontenc}
  % \usepackage[sfdefault]{noto} % use noto google font
\fi

% enable mathematical syntax for some symbols like \varnothing
\usepackage{amssymb}

% bubble diagram configuration
\usepackage{smartdiagram}
\smartdiagramset{
  % defaut font size is \large, so adjust to harmonize with sidebar layout
  bubble center node font = \footnotesize,
  bubble node font = \footnotesize,
  % default: 4cm/2.5cm; make minimum diameter relative to sidebar size
  bubble center node size = 0.4\sidebartextwidth,
  bubble node size = 0.25\sidebartextwidth,
  distance center/other bubbles = 1.5em,
  % set center bubble color
  bubble center node color = maincolor!70,
  % define the list of colors usable in the diagram
  set color list = {maincolor!10, maincolor!40,
  maincolor!20, maincolor!60, maincolor!35},
  % sets the opacity at which the bubbles are shown
  bubble fill opacity = 0.8,
}

\titlespacing{\section}{0pc}{2pt}{1pt}
\titlespacing{\subsection}{0pc}{1pc}{0pc}
%-------------------------------------------------------------------------------
%                            PERSONAL INFORMATION
%-------------------------------------------------------------------------------
% profile picture
\cvprofilepic{./photo_cv.jpg}
% your name
\cvname{Clément Hottier}
% job title/career
\cvjobtitle{Data Scientist}
% date of birth
\cvbirthday{22 Juin 1992}
% phone number
\cvphone{+33 6 60 35 56 85}
% personal website
% \cvsite{https://pandascience.net}
% email address
% \cvmail{clement.hottier@obspm.fr}
\cvmail{hottierclement@gmail.com}
% pgp key
% add additional information
% \newcommand{\additional}{some more?}


%-------------------------------------------------------------------------------
%                              SIDEBAR 1st PAGE
%-------------------------------------------------------------------------------
% overwrite default icons and order of personal information
\renewcommand{\personaltable}{%
  \begin{personal}
    \circleicon{\faInfo}     & \cvbirthday \\
    \circleicon{\faAt}       & \cvmail \\
    % \circleicon{\faGlobe}    & \cvsite \\
    \circleicon{\faPhone}    & \cvphone \\
    % add another line
  \end{personal}
}

% add more profile sections to sidebar on first page
\addtofrontsidebar{
  % include gosquare national flags from https://github.com/gosquared/flags;
  % naming according to ISO 3166-1 alpha-2 country codes
  \graphicspath{{pics/flags/}}

  \profilesection{Langue}
  \pointskill{\flag{FR.png}}{Français}{5}
  \pointskill{\flag{GB.png}}{Anglais}{4}
  \pointskill{\flag{ES.png}}{Espagnol}{1}

  \profilesection{Programmation}
  \barskill{}{ADQL/SQL}{85}
  \barskill{}{Python : numpy/scipy/pandas}{95}
  \barskill{}{Java}{90}
  \barskill{}{Fortran/C}{70}
  \barskill{}{Bash}{85}
  \barskill{}{Cluster computing (slurm)}{95}
  \barskill{}{Git}{75}
  \barskill{}{Docker}{60}

}

%-------------------------------------------------------------------------------
%                         TABLE ENTRIES RIGHT COLUMN
%-------------------------------------------------------------------------------
\begin{document}

\makefrontsidebar

\cvsection{Expériences Professionnelles}
\begin{cvtable}
  \cvitem{2019-2020}{Enseignant Chercheur (ATER)}{Observatoire de Paris}{- Exploration
    de données et identification croisée de structures\\
    - Big Data\\
    - Distribution de base de données\\
    - Administration de données non réduite (HDF5)\\
    - Implémentation d'algorithme de classification et création de catalogues
    }
    \cvitem{2016-2019}{Doctorant (Avec mission d'enseignement)}{Observatoire de Paris}{
    - Conception et mise au point algorithme d'analyse par déconvolution Bayésienne\\
    - Implémentation et Intégration continue d'un logiciel d'analyse de données (Git et
    GitLab CI)\\
    - Gestion d'un logiciel d'analyse en mode production dans un centre de calcul ($\approx
  200\.000$hCPU) et des sorties brut ($10$TB)\\
   - Développement du Pipeline de post processing et de visualisation 
    
}

  \cvitem{Mars - Juillet 2016}{Stage Master 2}{Institut d'Astrophysique de Paris}
  {Conception et mise au point d'un modèle d'analyse par maximisation de vraissemblance et
  chaine de Markov}
  % \cvitem{Avril - Juillet 2015}{Stage M1}{Observatoire de Paris - LERMA}
  % {Étude de l'impact des Quasars radio loud sur le milieu intergalactique à l'époque de la
  % réionisation}
\end{cvtable}

\cvsection{Expériences d'enseignement}
\subsection{Universitaire}
\begin{cvtable}
  \cvitem{2016-2020}{Cours Magistraux et Travaux dirigés}{Observatoire de Paris}{
  Programmation Scientifique - Programmation Latex - Mécanique Quantique}
  
  \cvitem{2016-2020}{Stage d'observation}{Observatoire de Paris/Univ Paris-Saclay}
  {- Encadrement d'observations sur instruments professionnels\\
   - Développement de la boîte à outils de réduction des données et distribution en package
 python et conteneur Docker}
\end{cvtable}

\subsection{Médiation scientifique}
\begin{cvtable}
  \cvitem{2018}
  {Conférence Pint of science}
  {\href{https://pintofscience.fr/}{\underline{Pint of science France}}}{La mission spatiale Gaia}
  \cvitem{2019}{Conférence étoiles en
  scène}{\href{https://www.obspm.fr/etoiles-en-scene-au-grand-rex.html}{\underline{Com Obs. Paris}}}{Conférence express au Grand Rex}
\end{cvtable}


\cvsection{Communications}
\subsection{Publications}
\begin{cvtable}
  \cvitem{2020}
  {FEDReD II : 3D Extinction Map with 2MASS and Gaia DR2 data}
  {\href{https://ui.adsabs.harvard.edu/abs/2020arXiv200703734H/abstract}{\underline{sous
  presse}}}
  {}

  \cvitem{2020}
  {FEDReD I: 3D dust and stellar maps by \\Bayesian deconvolution}
  {\href{https://ui.adsabs.harvard.edu/abs/2020arXiv200704455B/abstract}{\underline{sous presse}}}
  {}
\end{cvtable}

\subsection{Conférences Internationales}
\begin{cvtable}
  \cvitem{Avril 2019}
  {Dust distribution in the Milky Way}
  {\href{https://ui.adsabs.harvard.edu/abs/2019gaia.confE..55H/abstract}{\underline{$53^{rd}$ ESLAB}}}
  {}

  \cvitem{Juillet 2018}
  {Dust and star distribution in the Milky Way disk}
  {\href{https://ui.adsabs.harvard.edu/\#abs/2018sf2a.conf..345H/abstract}{\underline{SF2A Proceeding}}}
  {}
\end{cvtable}


\cvsection{Responsabilités Administratives}
\begin{cvtable}
  \cvitemshort{2017-2019}{Représentant des étudiants de l'Observatoire de
  Paris au conseil d'administration (CA), au conseil de l'Unité Formation-Enseignement (CUFE)
et au conseil de la documentation (CDOC)}

  \cvitemshort{2019}{Membre du comité de pilotage pour la réforme du Master 1 SUTS de
l'Observatoire de Paris}
\end{cvtable}

\cvsection{Formations}
\begin{cvtable}
  \cvitem{2016-2019}{\textbf{Doctorat} (\href{https://tel.archives-ouvertes.fr/tel-02879238/document}{Manuscrit
  en ligne})}{Observatoire de Paris}{}
  \cvitem{2014-2016}{Master}{Observatoire de Paris}{Dynamique des systèmes gravitationnels}
  % \cvitem{2014-2015}{Master 1}{Observatoire de Paris}{Sciences de L'univers et technologies spatiales}
  % \cvitem{2011-2014}{Licence}{Université Lille 1}{Physique fondamentale}
\end{cvtable}
\end{document} 
