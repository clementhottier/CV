% FortySecondsCV LaTeX template
% Copyright © 2019 René Wirnata <rene.wirnata@pandascience.net>
% Licensed under the 3-Clause BSD License. See LICENSE file for details.
%
% Attributions
% ------------
% * fortysecondscv is based on the twentysecondcv class by Carmine Spagnuolo 
%   (cspagnuolo@unisa.it), released under the MIT license and available under
%   https://github.com/spagnuolocarmine/TwentySecondsCurriculumVitae-LaTex
% * further attributions are indicated immediately before corresponding code


%-------------------------------------------------------------------------------
%                             ADDITIONAL PACKAGES
%-------------------------------------------------------------------------------
\documentclass[
 a4paper, 10.0pt,
%   showframes,
%   maincolor=cvgreen,
%   sectioncolor=red,
%   subsectioncolor=orange
  sidebarwidth=0.32\paperwidth,
%   topbottommargin=0.03\paperheight,
%   leftrightmargin=20pt
]{fortysecondscv}

% improve word spacing and hyphenation
\usepackage{microtype}
\usepackage{ragged2e}

% take care of proper font encoding
\ifxetex
  \usepackage{fontspec}
  \defaultfontfeatures{Ligatures=TeX}
  % \newfontfamily\headingfont[Path = fonts/]{segoeuib.ttf} % local font
\else
  \usepackage[utf8]{inputenc}
  \usepackage[T1]{fontenc}
  % \usepackage[sfdefault]{noto} % use noto google font
\fi

% enable mathematical syntax for some symbols like \varnothing
\usepackage{amssymb}

% bubble diagram configuration
\usepackage{smartdiagram}
\smartdiagramset{
  % defaut font size is \large, so adjust to harmonize with sidebar layout
  bubble center node font = \footnotesize,
  bubble node font = \footnotesize,
  % default: 4cm/2.5cm; make minimum diameter relative to sidebar size
  bubble center node size = 0.4\sidebartextwidth,
  bubble node size = 0.25\sidebartextwidth,
  distance center/other bubbles = 1.5em,
  % set center bubble color
  bubble center node color = maincolor!70,
  % define the list of colors usable in the diagram
  set color list = {maincolor!10, maincolor!40,
  maincolor!20, maincolor!60, maincolor!35},
  % sets the opacity at which the bubbles are shown
  bubble fill opacity = 0.8,
}

\titlespacing{\section}{0pc}{2pt}{1pt}
\titlespacing{\subsection}{0pc}{1pc}{0pc}
%-------------------------------------------------------------------------------
%                            PERSONAL INFORMATION
%-------------------------------------------------------------------------------
% profile picture
\cvprofilepic{./photo_cv.jpg}
% your name
\cvname{Clément Hottier}
% job title/career
\cvjobtitle{Post Doc/Assistant Professor}
% date of birth
\cvbirthday{22 June 1992}
% phone number
\cvphone{+33 1 45 07 79 07}
% personal website
% \cvsite{https://pandascience.net}
% email address
\cvmail{clement.hottier@obspm.fr}
% pgp key
% add additional information
% \newcommand{\additional}{some more?}


%-------------------------------------------------------------------------------
%                              SIDEBAR 1st PAGE
%-------------------------------------------------------------------------------
% overwrite default icons and order of personal information
\renewcommand{\personaltable}{%
  \begin{personal}
    \circleicon{\faInfo}     & \cvbirthday \\
    \circleicon{\faAt}       & \cvmail \\
    % \circleicon{\faGlobe}    & \cvsite \\
    \circleicon{\faPhone}    & \cvphone \\
    % add another line
  \end{personal}
}

% add more profile sections to sidebar on first page
\addtofrontsidebar{
  % include gosquare national flags from https://github.com/gosquared/flags;
  % naming according to ISO 3166-1 alpha-2 country codes
  \graphicspath{{pics/flags/}}

  \profilesection{Languages}
  \pointskill{\flag{FR.png}}{French}{5}
  \pointskill{\flag{GB.png}}{English}{4}
  \pointskill{\flag{ES.png}}{Spanish}{1}

  \profilesection{Computing skills}
  \barskill{}{ADQL/SQL}{85}
  \barskill{}{Python/numpy/scipy/pandas}{95}
  \barskill{}{Java}{90}
  \barskill{}{Fortran/C}{65}
  \barskill{}{bash}{80}
  \barskill{}{Cluster use (slurm)}{95}

}

%-------------------------------------------------------------------------------
%                         TABLE ENTRIES RIGHT COLUMN
%-------------------------------------------------------------------------------
\begin{document}

\makefrontsidebar

\cvsection{Research experiences}
\begin{cvtable}
  \cvitem{oct 2019- sep 2020}{Post Doc/Assistant Professor}{Observatoire de Paris - GEPI}{Study of the Milky Way disk
  with Gaia 2MASS, UKIDSS and VVV datas.}
  \cvitem{2016-2019}{PhD Thesis}{Observatoire de Paris - GEPI}{Study of the Milky Way disk
  with Gaia and 2MASS.}
  \cvitem{Mar - Jul 2016}{Master's degree intern.}{Institut d'Astrophysique de Paris}
  {Modeling non-spherical galaxy clusters by the Metropolis-Hasting algorithme. Supervised by
  Mamon, G.}
  \cvitem{Apr - Jul 2015}{Master's degree 1$^{th}$ year intern.}{Observatoire de Paris - LERMA}
  {Study of the 21cm forest signal at the Epoch of Reionization and the imprint of quasars}
\end{cvtable}


\cvsection{Education}
\begin{cvtable}
  \cvitem{sep 2016 - oct 2019}{\textbf{PhD Thesis}}{Observatoire de Paris - Gepi}{"3D
  distribution of star and dust in the Milky Way disk", supervised by Babusiaux, C. and Arenou,
F. Defend the 17$^{th}$ septembre 2019}
  %mettre le titre de la thèse et la licence
  \cvitem{2014-2016}{Master's degree}{Observatoire de Paris}{"Astronomy and Astrophysics"}
  % \cvitem{2014-2015}{Master 1}{Observatoire de Paris}{Sciences de L'univers et technologies spatiales}
  % \cvitem{2011-2014}{Licence}{Université Lille 1}{Physique fondamentale}
\end{cvtable}

\cvsection{Communications}
\subsection{Publications}
\begin{cvtable}
  \cvitem{2020}
  {FEDReD II : 3D Extinction Map\\ with 2MASS and Gaia DR2 data}
  {Submitted : A\&A}
  {Hottier \textit{et.al.}}

  \cvitem{2020}
  {FEDReD I: 3D dust and stellar maps by \\Bayesian deconvolution}
  {Submitted : A\&A}
  {Babusiaux \textit{et.al.}}

  \cvitem{2019}
  {Gaia-2MASS 3D maps of Galactic interstellar dust\\within 3 kpc}
  {\href{https://ui.adsabs.harvard.edu/\#abs/2019arXiv190204116L/abstract}{\underline{A\&A}}}
  {Lallement \textit{et. al.}}

  \cvitem{2018}
  {Imprints of quasar duty cycle on the 21-cm signal from the Epoch of Reionization}
  {\href{https://ui.adsabs.harvard.edu/\#abs/2019arXiv190204116L/abstract}{\underline{MNRAS}}}
  {Bolgar \textit{et. al.}}
\end{cvtable}

\subsection{Talk}
\begin{cvtable}
  \cvitem{Apr 2019}
  {Dust distribution in the Milky Way}
  {\href{https://ui.adsabs.harvard.edu/abs/2019gaia.confE..55H/abstract}{\underline{$53^{rd}$ ESLAB}}}
  {Hottier \textit{et. al.}}

  \cvitem{Jul 2018}
  {Dust and star distribution in the Milky Way disk}
  {\href{https://ui.adsabs.harvard.edu/\#abs/2018sf2a.conf..345H/abstract}{\underline{SF2A Proceeding}}}
  {Hottier \textit{et. al.}}
\end{cvtable}

\cvsection{Teaching}
\subsection{University}
\begin{cvtable}
  \cvitem{2016-2019}{Tutorials}{Masters degree Obs. Paris}{Quantum Mechanics  -
  Programming}
  
  \cvitem{2016-2020}{Lectures}{Masters degree Obs. Paris} {Programming}

  \cvitem{2016-2020}{Observational project}{Obs. Paris}{At the Observatoire de Paris and at the
    Observatoire de Haute Provence
  }
\end{cvtable}

\subsection{Scientific Mediation}
\begin{cvtable}
  \cvitem{2018}
  {Pint of Science Conference}
  {\href{https://pintofscience.fr/}{\underline{Pint of science France}}}{"The Gaia Mission"}
  \cvitem{2019}{Stars on
  stage}{\href{https://youtu.be/zViGxRlhsqs}{\underline{Obs.
Paris}}}{Express talk (Ted Talk like) at Grand Rex movie theatre}

  \cvitem{2016-2020}{International Earth Science Olympiad}
  {Science a l'école/Obs. Paris}
  {Astronomy Coach and Mentor of the french national delegation}
\end{cvtable}

\cvsection{Administrative Responsibility}
\begin{cvtable}
  \cvitemshort{2017-2019}{Student representative at Executive Board, Teaching Board and Library
  Board of the Observatoire de Paris }

  \cvitemshort{2019}{Steering group member of the new Observatoire Paris Master degree.}
\end{cvtable}

\end{document} 
